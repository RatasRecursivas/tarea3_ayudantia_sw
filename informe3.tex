\documentclass[a4paper,11pt]{report}
%
%--------------------   start of the 'preamble'
%
\usepackage{graphicx,amssymb,amstext,amsmath}
\usepackage{enumerate}
\usepackage{hyperref}
\usepackage[spanish]{babel}
\usepackage[utf8]{inputenc}
\hypersetup{
    colorlinks,
    citecolor=black,
    filecolor=black,
    linkcolor=black,
    urlcolor=black
}
%
%%    homebrew commands -- to save typing
\newcommand\etc{\textsl{etc}}
\newcommand\eg{\textsl{eg.}\ }
\newcommand\etal{\textsl{et al.}}
\newcommand\Quote[1]{\lq\textsl{#1}\rq}
\newcommand\fr[2]{{\textstyle\frac{#1}{#2}}}
\newcommand\miktex{\textsl{MikTeX}}
\newcommand\comp{\textsl{The Companion}}
\newcommand\nss{\textsl{Not so Short}}
%
%---------------------   end of the 'preamble'
%
\begin{document}
%-----------------------------------------------------------
\title{Patrones de Diseño}
\author{Patricio Pérez\\
        Sebastián Rocha\\
        Natalia Tarifeño}
\maketitle

%-----------------------------------------------------------
\title {\textbf{PATRONES DE DISEÑO}}\\

A diario se nos presentan diversos problemas  que debemos enfrentar
y/o solucionar,
muchos de estos problemas tienen características muy similares entre sí, a las cuales ya se les han presentado una solución.\\
Estas soluciones ya desarrolladas y probadas exitosamente son los patrones de diseños,  las
cuales buscan de una manera simple, consistente, efectiva y correcta darnos esta respuesta, evitando errores comunes ya abordados.\\
Una de sus características principales es el ahorro de tiempo al momento de abordar un problema que ya tiene solución.\\
Actualmente existen una gran variedad de patrones de diseños, dentro
de los cuales destacan: singleton, proxy, facade o Modelo Vista Controlador (MVC), Modelo
Vista Template (MVT) , entre otros.\\
En particular nos centraremos en el patrón Modelo Vista Controlador comúnmente llamado MVC por sus siglas.\\

\title { \textbf{MODELO VISTA CONTROLADOR (MVC)}}\\

El modelo vista controlador, separa la lógica de negocios con la interfaz de usuario. Aquí se distinguen 3 capas:

\begin{itemize}
    \item{Modelo}
    \item{Vista}
    \item{Controlador}
\end{itemize}

\begin{figure}[!ht]
\begin{center}
  \includegraphics[width=0.9\textwidth]{mvc.jpg}
  \caption{MVC aplicado en un software}
\end{center}
\end{figure}

